\documentclass[11pt]{beamer}
\usetheme{Dresden}
\usepackage[utf8]{inputenc}
\usepackage{amsmath}
\usepackage{amsfonts}
\usepackage{amssymb}
\usepackage[backend=bibtex]{biblatex}
\usepackage{cancel}
\usepackage{hyperref}
\usepackage{minted}
\addbibresource{talkslides.bib}

\author{Mitchell Riley}
\title{Exact Real Arithmetic in Haskell}
\date{12th May 2015}
\beamertemplatenavigationsymbolsempty
\setbeamertemplate{footline}{}
\begin{document}

\begin{frame}
\titlepage
\end{frame}

\section{Intro}

\begin{frame}
\frametitle{Numbers}
Natural numbers: Counting numbers $\{0, 1, 2, \dots\}$
\\~\\
Integers: Natural numbers and negatives $\{\dots, -2, -1, 0, 1, 2, \dots\}$
\\~\\
Rational: Fractions $\{\frac{p}{q} : p, q \text{ are integers}\}$
\\~\\
Reals: All values on the continuum
\end{frame}

\begin{frame}
\frametitle{Floating point}
\begin{align*}
\begin{pmatrix}
64919121   & -159018721 \\
41869520.5 & -102558961
\end{pmatrix} x =
\begin{pmatrix}
1 \\
0
\end{pmatrix}
\end{align*}
\end{frame}

\begin{frame}[fragile]
\begin{minted}{haskell}
solveAxb :: Fractional t =>
            (t, t, t, t) -> (t, t) -> (t, t)
solveAxb (a11, a12,
          a21, a22)
         (b1,
          b2)
  = (( a22 * b1 - a12 * b2) / det,
     (-a21 * b1 + a11 * b2) / det)
  where det = a11 * a22 - a12 * a21

a :: Fractional t => (t, t, t, t)
a = (64919121,   -159018721,
     41869520.5, -102558961)

b :: Fractional t => (t, t)
b = (1,
     0)
\end{minted}
\end{frame}

\begin{frame}
Using \texttt{Double}s:
\begin{align*}
x =
\begin{pmatrix}
102558961 \\
41869520.5
\end{pmatrix}
\end{align*}
\pause
Actually...
\begin{align*}
x =
\begin{pmatrix}
205117922 \\
83739041
\end{pmatrix}
\end{align*}
\end{frame}


\begin{frame}
\frametitle{Almost integers}
\texttt{Double}s:
\only<1>{\begin{align*}
\sin(2017 \sqrt[5]{2}) = -1
\end{align*}}
\only<2->{\begin{align*}\xcancel{
\sin(2017 \sqrt[5]{2}) = -1
}
\end{align*}}
\pause
Actually:
\begin{align*}
\sin(2017 \sqrt[5]{2}) = -0.9999999999999999785
\end{align*}
\end{frame}

\begin{frame}
\frametitle{Arbitrary precision arithmetic}
Arbitrary precision \emph{integer} arithmetic comes built in to
Haskell (and Python and Ruby and ...)
\\~\\
Can use this to implement arbitrary precision floating point, i.e. $n
\times b^c$
\\~\\
Doesn't save us with \texttt{sqrt}, \texttt{sin}, \texttt{pi} ...
\end{frame}

\begin{frame}
\frametitle{\emph{Exact} arithmetic}
Represents any (computable) real number \emph{exactly}
\\~\\
Transcendental functions in \texttt{Floating} are no longer approximations
\\~\\
We are able to request any output precision, and the details are
handled for us
\end{frame}

\section{Fast Cauchy}

\begin{frame}
\frametitle{Cauchy Sequences}
\begin{definition}
A \emph{Cauchy sequence} is a sequence of rational numbers $\{x_0, x_1, \dots, x_i, \dots\}$
such that for any $\epsilon$, there exists an $N$ such that
\begin{align*}
\vert x_n - x_m \vert < \epsilon
\end{align*}
for any $m > N, n > N$.
\end{definition}
~\\
The real numbers are \emph{defined} to be the set of all Cauchy
sequences (where we consider two sequences to be the same if their
difference converges to $0$)
\end{frame}

\begin{frame}
\begin{align*}
\frac{1}{3} &= \{\frac{1}{3}, \frac{1}{3}, \frac{1}{3}, \dots\} \\
~\\
\pi &= \{3, 3.1, 3.14, 3.141, \dots\}
\end{align*}
\end{frame}

\begin{frame}
\frametitle{Effective Cauchy}
\begin{definition}
A real number $x$ is represented as an \emph{effective Cauchy sequence} if
there is a sequence of rational numbers $\{x_0, x_1, \dots, x_i,
\dots\}$ such that
\begin{align*}
\vert x - x_p \vert < 2^{-p}
\end{align*}
\end{definition}
\begin{align*}
\frac{1}{3} &= \{\frac{0}{1}, \frac{1}{2}, \frac{1}{4}, \frac{3}{8}, \frac{5}{16}, \frac{11}{32}, \dots\}\\
~\\
\pi &= \{\frac{3}{1}, \frac{6}{2}, \frac{13}{4}, \frac{25}{8}, \frac{50}{16}, \frac{101}{32}, \dots\}
\end{align*}
\end{frame}

\begin{frame}[fragile]
\frametitle{Fast Binary Cauchy}
\begin{definition}
A real number $x$ is represented as a \emph{Fast Binary Cauchy Sequence} if
there is a sequence of \emph{integers} $\{n_0, n_1, \dots, n_i, \dots\}$ such that
\begin{align*}
\vert x - 2^{-p}n_p \vert < 2^{-p}
\end{align*}
\end{definition}
\begin{align*}
\frac{1}{3} &= \{0, 1, 1, 3, 5, 11, \dots \} \\
\pi &= \{3, 6, 13, 25, 50, 101, \dots \}
\end{align*}
\end{frame}

\begin{frame}[fragile]
\begin{minted}{haskell}
type CReal = Natural -> Integer
\end{minted}
  \begin{align*}
\frac{x[p] - 1}{2^p} < x < \frac{x[p] + 1}{2^p}
\end{align*}
\end{frame}

\begin{frame}[fragile]
\frametitle{Easy Stuff}
\begin{minted}{haskell}
fromInteger :: Integer -> CReal
fromInteger n = \p -> n * 2^p

negate :: CReal -> CReal
negate x = \p -> negate (x p)
\end{minted}
\end{frame}

\begin{frame}[fragile]
\frametitle{Addition}
If:
\begin{align*}
\frac{a[p+2] - 1}{2^{p+2}} &< a < \frac{a[p+2] + 1}{2^{p+2}} \\
\frac{b[p+2] - 1}{2^{p+2}} &< b < \frac{b[p+2] + 1}{2^{p+2}}
\end{align*}
then:
\begin{align*}
\frac{r - 1}{2^p} & < a + b < \frac{r + 1}{2^p}
\end{align*}
where:
\begin{align*}
r = \lfloor \frac{a[p+2]+b[p+2]}{4} \rceil
\end{align*}
\begin{minted}{haskell}
(+) :: CReal -> CReal -> CReal
a + b = \p -> round $ ((a (p+2) + b (p+2)) % 4
\end{minted}
\end{frame}

\begin{frame}[fragile]
\frametitle{Transcendental functions}
Evaluated using Taylor series:
\begin{align*}
e^x &= 1 + x + \frac{1}{2!}x^2 + \frac{1}{3!}x^3 + \dots \\
~\\
\sin x &= x - \frac{1}{3!}x^3 + \frac{1}{5!}x^5 - \dots
\end{align*}
If $\vert x \vert < 1$, then eventually the terms are very small
\end{frame}

\begin{frame}
\frametitle{Disadvantages}
Result arrives all at once
\\~\\
If we later need more precision, we have to start all over
\end{frame}

\section{Continued Fractions}

\begin{frame}
\frametitle{Decimal representation}
Consider $\pi$:
\only<1>{\begin{align*}
\pi = 3
\end{align*}}
\only<2>{\begin{align*}
\pi = 3 + 0.1415926\dots
\end{align*}}
\only<3>{\begin{align*}
\pi = 3 + \frac{1}{10}(1)
\end{align*}}
\only<4>{\begin{align*}
\pi = 3 + \frac{1}{10}(1 + 0.415926\dots)
\end{align*}}
\only<5>{\begin{align*}
\pi = 3 + \frac{1}{10}(1 + \frac{1}{10}(4))
\end{align*}}
\only<6>{\begin{align*}
\pi = 3 + \frac{1}{10}(1 + \frac{1}{10}(4 + 0.15926\dots))
\end{align*}}
\only<7>{\begin{align*}
\pi = 3 + \frac{1}{10}(1 + \frac{1}{10}(4 + \frac{1}{10}(1)))
\end{align*}}
\only<8>{\begin{align*}
\pi = 3 + \frac{1}{10}(1 + \frac{1}{10}(4 + \frac{1}{10}(1 + 0.5926\dots)))
\end{align*}}
\only<9>{\begin{align*}
\pi = 3 + \frac{1}{10}(1 + \frac{1}{10}(4 + \frac{1}{10}(1 + \frac{1}{10}(5 + \dots))))
\end{align*}}
\end{frame}

\begin{frame}[fragile]
\frametitle{Decimal representation}
\begin{minted}{haskell}
decimal :: RealFrac a => a -> [Integer]
decimal a = let d = floor a in
            d : decimal ((a - fromInteger d) * 10)

decimal (1%3) => [0,3,3,3,3,3,3,...
decimal pi    => [3,1,4,1,5,9,2,...
\end{minted}
% TODO: step eval
\end{frame}

\begin{frame}
\frametitle{Decimal representation}
Some problems:
\begin{align*}
\frac{1}{3} = 0.333333333333333333333333333333333333333333333333333333333333333333333333333333333333333333\dots
\end{align*}
\\~\\
Implementing \texttt{Floating} on a stream of decimal digits
would be nasty
\\~\\
Why the 10?!
\end{frame}

\begin{frame}
\frametitle{Continued fractions}
Consider $\pi$:
\only<1>{\begin{align*}
\pi = 3
\end{align*}}
\only<2>{\begin{align*}
\pi = 3 + 0.1415926\dots
\end{align*}}
\only<3>{\begin{align*}
\pi = 3 + \frac{1}{7}
\end{align*}}
\only<4>{\begin{align*}
\pi = 3 + \frac{1}{7 + 0.0625132\dots}
\end{align*}}
\only<5>{\begin{align*}
\pi = 3 + \frac{1}{7 + \frac{1}{15}}
\end{align*}}
\only<6>{\begin{align*}
\pi = 3 + \frac{1}{7 + \frac{1}{15 + 0.9965996}\dots}
\end{align*}}
\only<7>{\begin{align*}
\pi = 3 + \frac{1}{7 + \frac{1}{15 + \frac{1}{1}}}
\end{align*}}
\only<8>{\begin{align*}
\pi = 3 + \frac{1}{7 + \frac{1}{15 + \frac{1}{1 + 0.0034172\dots}}}
\end{align*}}
\only<9>{\begin{align*}
\pi = 3 + \frac{1}{7 + \frac{1}{15 + \frac{1}{1 + \frac{1}{292 + \dots}}}}
\end{align*}}
\end{frame}

\begin{frame}[fragile]
Let us write this as $\pi = [3,7,15,1,292,\dots]$
\\~\\
% TODO: step eval
\begin{minted}{haskell}
type CF = [Integer]

cf :: Fractional a => a -> CF
cf a | fromInteger (floor a) == a = [a]
cf a | otherwise = let d = floor a in
         d : cf (recip (a - fromInteger d))

cf (1%3) => [0,3]
cf pi    => [3,7,15,1,292,...
\end{minted}
Every real number has a (essentially) unique expansion
\\~\\
This expansion is finite when the number is rational
\end{frame}

\begin{frame}[fragile]
\frametitle{Arithmetic}
Let us consider functions of the form:
\begin{align*}
\frac{ax + b}{cx + d}
\end{align*}
with $a,b,c,d$ all integers.
\begin{minted}{haskell}
type Hom = (Integer, Integer,
            Integer, Integer)

hom :: Hom -> [Integer] -> [Integer]
\end{minted}
\end{frame}

\begin{frame}[fragile]
Let $x = [x_0, ...] = x_0 + \frac{1}{x'}$, then:
\begin{align*}
\frac{ax + b}{cx + d} &= \frac{a(x_0 + \frac{1}{x'}) + b}{c(x_0 + \frac{1}{x'}) + d} \\
\uncover<2->{&= \frac{a(x_0 + \frac{1}{x'}) + b}{c(x_0 + \frac{1}{x'}) + d} \\}
\uncover<3->{&= \frac{ax_0 + a\frac{1}{x'} + b}{cx_0 + c\frac{1}{x'} + d}\\}
\uncover<4->{&= \frac{ax_0x' + a + bx'}{cx_0x' + c + dx'}}\\
\uncover<5->{&= \frac{(ax_0 + b)x' + a}{(cx_0 + d)x' + c}}
\end{align*}
\end{frame}

\begin{frame}[fragile]
\begin{minted}{haskell}
absorb :: Hom -> Integer -> Hom
absorb (a, b
        c, d) x0 = (a*x0 + b, a,
                    c*x0 + d, c)

hom h (x0:rest) == hom (absorb h x0) rest
\end{minted}
\end{frame}

\begin{frame}[fragile]
Let $z = \frac{ax + b}{cx + d}$. As $x \in [0, \infty)$,
$z$ must lie between $\frac{a}{c}$ and $\frac{b}{d}$.
\\~\\
So if $\frac{a}{c}$ and $\frac{b}{d}$ have the same integer part $q$, we
know for sure that $z = [q,\dots]$.
\\~\\
\begin{minted}{haskell}
tryEmit :: Hom -> Maybe Integer
tryEmit (a, b
         c, d) = if c /= 0 && d /= 0 && r == s then
                   Just r
                 else
                   Nothing
  where r = a `quot` c
        s = b `quot` d
\end{minted}
\end{frame}

\begin{frame}[fragile]
Let $z = q + \frac{1}{z'}$, then:
\begin{align*}
z' &= (z - q)^{-1} \\
\uncover<2->{&= \left(\frac{ax + b}{cx + d} - q\right)^{-1} \\}
\uncover<3->{&= \left(\frac{(ax + b) - q(cx + d)}{cx + d}\right)^{-1} \\}
\uncover<4->{&= \left(\frac{(a - cq)x + (b - dq)}{cx + d}\right)^{-1} \\}
\uncover<5->{&= \frac{cx + d}{(a - cq)x + (b - dq)}}
\end{align*}
\end{frame}

\begin{frame}[fragile]
\begin{minted}{haskell}
emit :: Hom -> Integer -> Hom
emit (a, b
      c, d) q = (c,       d,
                 a - c*q, b - d*q)

hom h cf == q : hom (emit q h) cf
\end{minted}
\end{frame}

\begin{frame}[fragile]
\begin{minted}{haskell}
hom :: Hom -> CF -> CF
hom h (x:xs) = case tryEmit h of
                 Just d -> d : hom (emit h d) (x:xs)
                 Nothing ->    hom (absorb h x) xs

2 * pi == hom (2, 0,
               0, 1) pi
       == CF [6,3,1,1,7,2,146,3,6,1]
\end{minted}
\end{frame}

\begin{frame}[fragile]
To do arithmetic, repeat all of the above with
\begin{align*}
\frac{axy + bx + cy + d}{exy + fx + gy + h}
\end{align*}
\begin{minted}{haskell}
type Bihom = (Integer, Integer, Integer, Integer,
              Integer, Integer, Integer, Integer)

bihom :: Bihom -> CF -> CF -> CF
\end{minted}
Now we can absorb from either $x$ or $y$, and emit similar to before.
\end{frame}

\begin{frame}[fragile]
\begin{minted}{haskell}
(+) = bihom (0, 1, 1, 0,
             0, 0, 0, 1)
(-) = bihom (0, 1, -1, 0,
             0, 0,  0, 1)
(*) = bihom (1, 0, 0, 0,
             0, 0, 0, 1)
(/) = bihom (0, 1, 0, 0,
             0, 0, 1, 0)
\end{minted}
\end{frame}

\begin{frame}[fragile]
\frametitle{Transcendental functions}
\begin{align*}
e^x &= [1, \frac{1}{x}, -2, \frac{3}{x}, 2, \frac{5}{x}, -2, \dots] \\
~\\
\log x &= [0, \frac{1}{x-1}, \frac{2}{1}, \frac{3}{x-1}, \frac{2}{3}, \frac{5}{x-1}, \frac{2}{5}, \dots]
\end{align*}
\pause
\begin{minted}{haskell}
type Hom a = (a, a, a, a)

hom :: (Num a, ...) => Hom a -> [a] -> CF

cfcf :: [CF] -> CF
cfcf = hom (1, 0, 0, 1)
\end{minted}
\end{frame}

\begin{frame}
\frametitle{Disadvantages}
Much slower than Cauchy sequences
\end{frame}

\begin{frame}
\frametitle{Code}
Fast Binary Cauchy: \\ \url{http://hackage.haskell.org/package/numbers}
\\~\\
Continued Fractions: \\ \url{http://github.com/mvr/cf}
\end{frame}

\begin{frame}
\nocite{*}
\printbibliography
\end{frame}
\end{document}

%%% Local Variables:
%%% mode: latex
%%% TeX-master: t
%%% End:

\documentclass[11pt]{beamer}
\usetheme{Dresden}
\usepackage[utf8]{inputenc}
\usepackage{amsmath}
\usepackage{amsfonts}
\usepackage{amssymb}
\usepackage[backend=bibtex]{biblatex}
\usepackage{cancel}
\usepackage{minted}
\addbibresource{talkslides.bib}

\author{Mitchell Riley}
\title{Exact Real Arithmetic in Haskell}
\date{12th May 2015}
\beamertemplatenavigationsymbolsempty
\setbeamertemplate{footline}{}
\begin{document}

\begin{frame}
\titlepage
\end{frame}

\section{Intro}

\begin{frame}
\frametitle{Floating point}
% I'm sure I don't need to tell a room full of programmers how
% badly behaved floating point numbers can be.
\begin{align*}
\begin{pmatrix}
64919121   & -159018721 \\
41869520.5 & -102558961
\end{pmatrix} x =
\begin{pmatrix}
1 \\
0
\end{pmatrix}
\end{align*}
\pause
Using \texttt{Double}s:
\only<2>{\begin{align*}
x =
\begin{pmatrix}
102558961 \\
41869520.5
\end{pmatrix}
\end{align*}}
\pause
\begin{align*}
\xcancel{x =
\begin{pmatrix}
102558961 \\
41869520.5
\end{pmatrix}}
\end{align*}
\pause
Actually...
\begin{align*}
x =
\begin{pmatrix}
205117922 \\
83739041
\end{pmatrix}
\end{align*}
\end{frame}


\begin{frame}
\frametitle{Almost integers}
\texttt{Double}s:
\only<1>{\begin{align*}
\sin(2017 \sqrt[5]{2}) = -1
\end{align*}}
\only<2->{\begin{align*}\xcancel{
\sin(2017 \sqrt[5]{2}) = -1
}
\end{align*}}
\pause
Actually:
\begin{align*}
\sin(2017 \sqrt[5]{2}) = -0.9999999999999999785
\end{align*}
\end{frame}

\begin{frame}
\frametitle{Arbitrary precision arithmetic}
Arbitrary precision \emph{integer} arithmetic comes built in to
Haskell (and Python and Ruby and ...)
\\
Can use this to implement arbitrary precision floating point.
\\
Doesn't save us with \texttt{sqrt}, \texttt{sin}, \texttt{pi} ...
\end{frame}

\begin{frame}
\frametitle{\emph{Exact} arithmetic}
Represents any (computable) real number \emph{exactly}.
\\
Functions in \texttt{Floating} are no longer approximations.
\\
We are able to request any output precision, and the details are
handled for us.
\end{frame}

%\section{Fast Cauchy}

\section{Continued Fractions}

\begin{frame}
\frametitle{Decimal representation}
Consider $\pi$:
\begin{align*}
\onslide<1->{\pi &= 3.1415\dots} \\
\pause
\onslide<2->{&= 3 + \frac{1}{10^1} + \frac{4}{10^2} + \frac{1}{10^3} + \frac{5}{10^4} + \dots}
\end{align*}
\end{frame}

\begin{frame}
\frametitle{Decimal representation}
Some problems:
\begin{align*}
\frac{1}{3} = 0.333333333333333333333333333333333333333333333333333333333333333333333333333333333333333333\dots
\end{align*}
\pause
Implementing \texttt{Floating} on a stream of decimal digits
would be nasty
\pause
Why the 10?!
\end{frame}

\begin{frame}
\frametitle{Continued fractions}
Consider $\pi$:
\only<1>{\begin{align*}
\pi = 3
\end{align*}}
\only<2>{\begin{align*}
\pi = 3 + 0.1415926\dots
\end{align*}}
\only<3>{\begin{align*}
\pi = 3 + \frac{1}{7}
\end{align*}}
\only<4>{\begin{align*}
\pi = 3 + \frac{1}{7 + 0.0625132\dots}
\end{align*}}
\only<5>{\begin{align*}
\pi = 3 + \frac{1}{7 + \frac{1}{15}}
\end{align*}}
\only<6>{\begin{align*}
\pi = 3 + \frac{1}{7 + \frac{1}{15 + 0.9965996}\dots}
\end{align*}}
\only<7>{\begin{align*}
\pi = 3 + \frac{1}{7 + \frac{1}{15 + \frac{1}{1}}}
\end{align*}}
\only<8>{\begin{align*}
\pi = 3 + \frac{1}{7 + \frac{1}{15 + \frac{1}{1 + 0.0034172\dots}}}
\end{align*}}
\only<9>{\begin{align*}
\pi = 3 + \frac{1}{7 + \frac{1}{15 + \frac{1}{1 + \frac{1}{292 + \dots}}}}
\end{align*}}
\end{frame}

\begin{frame}[fragile]
\frametitle{Continued fractions}

Let us write this as $\pi = [3,7,15,1,292,\dots]$
\\
\begin{minted}{haskell}
continuedFraction :: Fractional a => a -> [Integer]
continuedFraction = undefined
\end{minted}
\end{frame}

\begin{frame}
\nocite{*}
\printbibliography
\end{frame}
\end{document}

%%% Local Variables:
%%% mode: latex
%%% TeX-master: t
%%% End:

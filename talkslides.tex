\documentclass[11pt]{beamer}
\usetheme{Dresden}
\usepackage[utf8]{inputenc}
\usepackage{amsmath}
\usepackage{amsfonts}
\usepackage{amssymb}
\usepackage[backend=bibtex]{biblatex}
\addbibresource{talkslides.bib}

\author{Mitchell Riley}
\title{Exact Real Arithmetic in Haskell}
\date{12th May 2015}
\beamertemplatenavigationsymbolsempty
\setbeamertemplate{footline}{}
\begin{document}

\begin{frame}
\titlepage
\end{frame}

\begin{frame}
\frametitle{Floating point}
% I'm sure I don't need to tell a room full of programmers how
% badly behaved floating point numbers can be.
\begin{align*}
\begin{pmatrix}
64919121   & -159018721 \\
41869520.5 & -102558961
\end{pmatrix} x =
\begin{pmatrix}
1 \\
0
\end{pmatrix}
\end{align*}
\pause
Using \texttt{Double}s:
\begin{align*}
x =
\begin{pmatrix}
102558961 \\
41869520.5
\end{pmatrix}
\end{align*}
\pause
Actually...
\begin{align*}
x =
\begin{pmatrix}
205117922 \\
83739041
\end{pmatrix}
\end{align*}
\end{frame}

\begin{frame}
\nocite{*}
\printbibliography
\end{frame}
\end{document}
